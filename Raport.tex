% Options for packages loaded elsewhere
\PassOptionsToPackage{unicode}{hyperref}
\PassOptionsToPackage{hyphens}{url}
%
\documentclass[
]{article}
\usepackage{amsmath,amssymb}
\usepackage{lmodern}
\usepackage{iftex}
\ifPDFTeX
  \usepackage[T1]{fontenc}
  \usepackage[utf8]{inputenc}
  \usepackage{textcomp} % provide euro and other symbols
\else % if luatex or xetex
  \usepackage{unicode-math}
  \defaultfontfeatures{Scale=MatchLowercase}
  \defaultfontfeatures[\rmfamily]{Ligatures=TeX,Scale=1}
\fi
% Use upquote if available, for straight quotes in verbatim environments
\IfFileExists{upquote.sty}{\usepackage{upquote}}{}
\IfFileExists{microtype.sty}{% use microtype if available
  \usepackage[]{microtype}
  \UseMicrotypeSet[protrusion]{basicmath} % disable protrusion for tt fonts
}{}
\makeatletter
\@ifundefined{KOMAClassName}{% if non-KOMA class
  \IfFileExists{parskip.sty}{%
    \usepackage{parskip}
  }{% else
    \setlength{\parindent}{0pt}
    \setlength{\parskip}{6pt plus 2pt minus 1pt}}
}{% if KOMA class
  \KOMAoptions{parskip=half}}
\makeatother
\usepackage{xcolor}
\usepackage[margin=1in]{geometry}
\usepackage{longtable,booktabs,array}
\usepackage{calc} % for calculating minipage widths
% Correct order of tables after \paragraph or \subparagraph
\usepackage{etoolbox}
\makeatletter
\patchcmd\longtable{\par}{\if@noskipsec\mbox{}\fi\par}{}{}
\makeatother
% Allow footnotes in longtable head/foot
\IfFileExists{footnotehyper.sty}{\usepackage{footnotehyper}}{\usepackage{footnote}}
\makesavenoteenv{longtable}
\usepackage{graphicx}
\makeatletter
\def\maxwidth{\ifdim\Gin@nat@width>\linewidth\linewidth\else\Gin@nat@width\fi}
\def\maxheight{\ifdim\Gin@nat@height>\textheight\textheight\else\Gin@nat@height\fi}
\makeatother
% Scale images if necessary, so that they will not overflow the page
% margins by default, and it is still possible to overwrite the defaults
% using explicit options in \includegraphics[width, height, ...]{}
\setkeys{Gin}{width=\maxwidth,height=\maxheight,keepaspectratio}
% Set default figure placement to htbp
\makeatletter
\def\fps@figure{htbp}
\makeatother
\setlength{\emergencystretch}{3em} % prevent overfull lines
\providecommand{\tightlist}{%
  \setlength{\itemsep}{0pt}\setlength{\parskip}{0pt}}
\setcounter{secnumdepth}{-\maxdimen} % remove section numbering
\usepackage{polski}
\usepackage{mathtools}
\usepackage{amsthm}
\usepackage{amssymb}
\usepackage{icomma}
\usepackage{upgreek}
\usepackage{xfrac}
\usepackage{scrextend}
\usepackage{float}
\usepackage{tabularx}
\usepackage{hyperref}
\usepackage{caption}
\usepackage{enumitem}
\ifLuaTeX
  \usepackage{selnolig}  % disable illegal ligatures
\fi
\IfFileExists{bookmark.sty}{\usepackage{bookmark}}{\usepackage{hyperref}}
\IfFileExists{xurl.sty}{\usepackage{xurl}}{} % add URL line breaks if available
\urlstyle{same} % disable monospaced font for URLs
\hypersetup{
  pdftitle={Co wpływa na długość naszego życia?},
  pdfauthor={Szymon Malec, Michał Wiktorowski},
  hidelinks,
  pdfcreator={LaTeX via pandoc}}

\title{Co wpływa na długość naszego życia?}
\author{Szymon Malec, Michał Wiktorowski}
\date{}

\begin{document}
\maketitle

\raggedbottom

\hypertarget{wstux119p}{%
\section{Wstęp}\label{wstux119p}}

~~~~~W niniejszej pracy chcielibyśmy przeanalizować, jak poszczególne
czynniki zewnętrzne wpływają na długość naszego życia. Jak się okazuje,
jedne z nich są poważnym problemem, podczas gdy inne w niewielkim
stopniu przyczyniają się na długość życia. Do analizy posłużymy się
zbiorem danych charakteryzującym wiele państw świata pod kilkoma
aspektami, między innymi takimi jak jakość edukacji, spożycie alkoholu,
czy zapadalność na poszczególne choroby. Link do danych znajduje się
\href{https://www.kaggle.com/datasets/kumarajarshi/life-expectancy-who?fbclid=IwAR2HtwUPyioM4tHmuae7B2owTUB8q3XlmpP12LbTM9NYDsi4qtaWGOYoNDE}{\(\color{blue}{\text{tutaj}}\)}.

\hypertarget{opis-danych}{%
\section{Opis danych}\label{opis-danych}}

\hypertarget{braki-w-danych}{%
\section{Braki w danych}\label{braki-w-danych}}

~~~~~Zanim rozpoczniemy analizować nasze dane, sprawdźmy, czy zawierają
one jakieś braki.

\begin{longtable}[]{@{}
  >{\raggedright\arraybackslash}p{(\columnwidth - 32\tabcolsep) * \real{0.2857}}
  >{\raggedleft\arraybackslash}p{(\columnwidth - 32\tabcolsep) * \real{0.0446}}
  >{\raggedleft\arraybackslash}p{(\columnwidth - 32\tabcolsep) * \real{0.0446}}
  >{\raggedleft\arraybackslash}p{(\columnwidth - 32\tabcolsep) * \real{0.0446}}
  >{\raggedleft\arraybackslash}p{(\columnwidth - 32\tabcolsep) * \real{0.0446}}
  >{\raggedleft\arraybackslash}p{(\columnwidth - 32\tabcolsep) * \real{0.0446}}
  >{\raggedleft\arraybackslash}p{(\columnwidth - 32\tabcolsep) * \real{0.0446}}
  >{\raggedleft\arraybackslash}p{(\columnwidth - 32\tabcolsep) * \real{0.0446}}
  >{\raggedleft\arraybackslash}p{(\columnwidth - 32\tabcolsep) * \real{0.0446}}
  >{\raggedleft\arraybackslash}p{(\columnwidth - 32\tabcolsep) * \real{0.0446}}
  >{\raggedleft\arraybackslash}p{(\columnwidth - 32\tabcolsep) * \real{0.0446}}
  >{\raggedleft\arraybackslash}p{(\columnwidth - 32\tabcolsep) * \real{0.0446}}
  >{\raggedleft\arraybackslash}p{(\columnwidth - 32\tabcolsep) * \real{0.0446}}
  >{\raggedleft\arraybackslash}p{(\columnwidth - 32\tabcolsep) * \real{0.0446}}
  >{\raggedleft\arraybackslash}p{(\columnwidth - 32\tabcolsep) * \real{0.0446}}
  >{\raggedleft\arraybackslash}p{(\columnwidth - 32\tabcolsep) * \real{0.0446}}
  >{\raggedleft\arraybackslash}p{(\columnwidth - 32\tabcolsep) * \real{0.0446}}@{}}
\toprule()
\endhead
year & 2000 & 2001 & 2002 & 2003 & 2004 & 2005 & 2006 & 2007 & 2008 &
2009 & 2010 & 2011 & 2012 & 2013 & 2014 & 2015 \\
life\_expectancy & 0 & 0 & 0 & 0 & 0 & 0 & 0 & 0 & 0 & 0 & 0 & 0 & 0 &
10 & 0 & 0 \\
adult\_mortality & 0 & 0 & 0 & 0 & 0 & 0 & 0 & 0 & 0 & 0 & 0 & 0 & 0 &
10 & 0 & 0 \\
infant\_deaths & 0 & 0 & 0 & 0 & 0 & 0 & 0 & 0 & 0 & 0 & 0 & 0 & 0 & 0 &
0 & 0 \\
alcohol & 1 & 1 & 1 & 1 & 1 & 2 & 1 & 1 & 1 & 1 & 1 & 1 & 1 & 2 & 1 &
177 \\
percentage\_expenditure & 0 & 0 & 0 & 0 & 0 & 0 & 0 & 0 & 0 & 0 & 0 & 0
& 0 & 0 & 0 & 0 \\
hepatitis\_B & 98 & 88 & 70 & 52 & 45 & 36 & 32 & 24 & 20 & 17 & 15 & 13
& 13 & 11 & 10 & 9 \\
measles & 0 & 0 & 0 & 0 & 0 & 0 & 0 & 0 & 0 & 0 & 0 & 0 & 0 & 0 & 0 &
0 \\
BMI & 2 & 2 & 2 & 2 & 2 & 2 & 2 & 2 & 2 & 2 & 2 & 2 & 2 & 4 & 2 & 2 \\
under\_five\_deaths & 0 & 0 & 0 & 0 & 0 & 0 & 0 & 0 & 0 & 0 & 0 & 0 & 0
& 0 & 0 & 0 \\
polio & 3 & 3 & 2 & 2 & 2 & 2 & 1 & 1 & 1 & 1 & 1 & 0 & 0 & 0 & 0 & 0 \\
total\_expenditure & 4 & 4 & 4 & 3 & 3 & 3 & 3 & 3 & 3 & 3 & 3 & 3 & 2 &
2 & 2 & 181 \\
diphtheria & 3 & 3 & 2 & 2 & 2 & 2 & 1 & 1 & 1 & 1 & 1 & 0 & 0 & 0 & 0 &
0 \\
HIV\_AIDS & 0 & 0 & 0 & 0 & 0 & 0 & 0 & 0 & 0 & 0 & 0 & 0 & 0 & 0 & 0 &
0 \\
GDP & 29 & 28 & 28 & 28 & 27 & 27 & 27 & 27 & 27 & 27 & 27 & 27 & 29 &
33 & 28 & 29 \\
population & 40 & 40 & 40 & 40 & 40 & 40 & 40 & 40 & 40 & 40 & 40 & 40 &
41 & 49 & 41 & 41 \\
thinness\_1\_19\_years & 2 & 2 & 2 & 2 & 2 & 2 & 2 & 2 & 2 & 2 & 2 & 2 &
2 & 4 & 2 & 2 \\
thinness\_5\_9\_years & 2 & 2 & 2 & 2 & 2 & 2 & 2 & 2 & 2 & 2 & 2 & 2 &
2 & 4 & 2 & 2 \\
income\_composition\_of\_resources & 10 & 10 & 10 & 10 & 10 & 10 & 10 &
10 & 10 & 10 & 10 & 10 & 10 & 17 & 10 & 10 \\
schooling & 10 & 10 & 10 & 10 & 10 & 10 & 10 & 10 & 10 & 10 & 10 & 10 &
10 & 13 & 10 & 10 \\
\bottomrule()
\end{longtable}

\hypertarget{devoloped-a-developing}{%
\section{Devoloped a developing}\label{devoloped-a-developing}}

\hypertarget{analiza}{%
\section{Analiza}\label{analiza}}

~~~~~Aby przeanalizować, co najbardziej wpływa na długość życia, musimy
sprawdzić czy poszczególne zbiory danych są od siebie zależne, czy
występuje między nimi pewna korelacja. W tym celu posłużymy się
wykresami punktowymi i na ich podstawie wybierzemy odpowiednie narzędzia
do badania korelacji.

\begin{figure}[H]

\begin{center}\includegraphics{Raport_files/figure-latex/unnamed-chunk-3-1} \end{center}
\caption{Zależność długości życia od poszczególnych czynników}
\end{figure}

Od razu możemy zauważyć, że rozrzut punktów na poszczególnych wykresach
jest bardzo zróżnicowany. Na jednych wydaje się on być całkowicie
losowy, natomiast na innych widać pewną zależność. Ze względu na to
zróżnicowanie, nie użyjemy do sprawdzenia korelacji współczynnika
Pearsona - sprawdza on jedynie zależność liniową. Można za to
wykorzystać współczynnik korelacji Spearmana, który sprawdza zależność
monotoniczną między danymi. Ze względu na silną korelację, poddamy
głębszej analizie dwa czynniki - długość edukacji, oraz produkt krajowy
brutto.

\hypertarget{edukacja}{%
\subsubsection{Edukacja}\label{edukacja}}

\hypertarget{pkb}{%
\subsubsection{PKB}\label{pkb}}

\hypertarget{polska}{%
\section{Polska}\label{polska}}

\hypertarget{jak-dux142ugo-ux17cyjux105-inni}{%
\section{Jak długo żyją inni?}\label{jak-dux142ugo-ux17cyjux105-inni}}

~~~~~Jak już wspomnieliśmy - długość życia zależy od wielu czynników i
dla każdego państwa na świecie inaczej się one charakteryzują. Rozważmy
na początku dane z 2015 roku - najnowsze, którymi dysponujemy. Średnia
długość życia na globie wyniosła wtedy 71,62 lata. Jednak rozrzut danych
dla poszczególnych państw jest duży. Mianem najdłużej żyjących ludzi
mogli się wtedy poszczycić Słoweńczycy z imponującą średnią życia aż 88
lat. Natomiast najkrócej żyjącymi ludźmi okazali się być obywatele
Sierra Leony z wynikiem zaledwie 51 lat. Polacy osiągneli wynik 77.5
lat, co stawia nas na 42 miejscu w rankingu - wynik dobry, ale nie
najlepszy.

\begin{figure}[H]

\begin{center}\includegraphics{Raport_files/figure-latex/unnamed-chunk-4-1} \end{center}
\caption{Wykresik}
\end{figure}

\hypertarget{podsumowanie}{%
\section{Podsumowanie}\label{podsumowanie}}

\end{document}
